\chapter{Conclusion}

With this project Bluetooth communication between an Android app and a FPGA based system accomplished. Visualisation of data transfer is realised using a LCD module. Different complex systems work together and demonstrate a proof of concept where an Android application can be used to manipulate and control FPGA based systems. More concrete example usage could be in image processing systems or industrial control applications. To reduce complexity and focus on concepts under study, scope of this work is kept rather narrow. However, technologies explored here show promising possibilities in areas like embedded development and internet of things.

In this study, code is written, compiled and used on MicroBlaze processor, however addition of a real-time operation system could be a further improvement on design level. Moreover it can enlarge the possible uses of this work. Another  possible extension could be developing functionalities of Console on Android app. This work provides synchronisation of 16 numbers, by extending the Console more flexibility and complexity can be achieved. 

As a result, this project shows promising methods on real-time control of FPGA based systems via an Android app. It provides a proof concept and demonstrates proposed concepts with an implementation. 
